\documentclass[tikz, border=1mm]{standalone}
% \usepackage[mode=buildnew]{standalone}% requires -shell-escape
\usepackage{tikz}
\usepackage{amsmath}
\usepackage{tikz-3dplot}
\usepackage{pgfplots}
\pgfplotsset{compat=newest}
\usetikzlibrary{arrows}
\usetikzlibrary{arrows.meta,decorations.markings}
\usepgfplotslibrary{fillbetween}

\tikzset{myarrowhead/.style={decoration={markings,mark=at position 1 with %
{\arrow[scale=1.,>={Triangle[length=8pt, width=7pt]}]{>}}},postaction={decorate}}}

\DeclareUnicodeCharacter{2212}{−}
\usepgfplotslibrary{groupplots,dateplot}
\usepgfplotslibrary{polar}
\usetikzlibrary{positioning}
\usetikzlibrary{patterns,shapes.arrows}
\pgfplotsset{compat=newest}

% Font
\usepackage{arev}

%%%%%%%%%%%%%%%%%%%%%%%%%%%%%%%%%%%%%%%%%%%%%%%%%%%%%%%%%%%%%%%%%%%%%%%%%%%%%%%%%%%%%%%%%%%%
\begin{document}

\pgfmathsetmacro{\marksize}{3}
\pgfmathsetmacro{\mylinewidth}{1.5pt}

%%%%%%%%%%%%%%%%%%%%%%%

\pgfmathsetmacro{\base}{-4.0/35.0}
\pgfmathsetmacro{\costoonehalfofheight}{0.86602540378443864676372317075}
\tikzset{declare function={upperboundofd1(\la)=-4/35+\la-\la^2;}}
\tikzset{declare function={radiusadmissible(\la)=(upperboundofd1(\la)-\base)/2;}}
\tikzset{declare function={upperboundofd1points(\la)=-4/35+(1+9/10)/2*(\la-\la^2);}}

%%%%%%%%%%%%%%%%%%%%%%%

\pgfplotsset{
	pointsinoverviews/.style={%
			thick, mark=*, mark size=3, mark options={solid}, only marks,
		}
}

%%%%%%%%%%%%%%%%%%%%%%%

\colorlet{ctop}{green}
\colorlet{ccenter}{blue}
\colorlet{cbottom}{red}

\begin{tikzpicture}
	\begin{axis}[
		width=0.9\linewidth,
		view={12}{10},
		hide axis,
		xmin=0.5, xmax=1.05,
		ymin=0, ymax=0.15,
		zmin=-0.115, zmax=0.15,
		y dir=reverse,
		]

		% ###########################
		% Orthotropic plane admissible with colored boundaries

		\addplot3[black!60!white, name path = upper, domain=1/2:1, y domain = 0:0,
 				line width=1.pt]
		(
		{x},
		{0},
		{upperboundofd1(x)}
		);

		\addplot3 [black!60!white, name path = lower, domain=1/2:1, y domain = 0:0, line width=1.pt]
		(
		{x},
		{0},
		{\base}
		);

		% Paths
		\addplot3[ctop, domain=1/2:5/6, y domain = 0:0,
 				line width=1.pt]
		(
		{x},
		{0},
		{upperboundofd1points(x)}
		);

		\addplot3 [cbottom, domain=1/2:5/6, y domain = 0:0, line width=1.pt]
		(
		{x},
		{0},
		{\base*(9/10)}
		);

		%%%%%%%%%%%%%%%%%%%%%%%%%%%%%%%%
		% Vertical lines

		\addplot3 [black!60!white,
 			mark=none,
			domain=\base/10*9:upperboundofd1points(3/6)),
			samples=60,samples y=0,
			line width=\mylinewidth,
			name path = zerothvertical
		]
		(
		{1/2},
		{0},
		{x}
		);

		\addplot3 [black!60!white, mark=none,
			domain=\base:upperboundofd1(4/6)),
			samples=60,samples y=0,
			line width=\mylinewidth
		]
		(
		{4/6},
		{0},
		{x}
		);

		\addplot3 [black!60!white, mark=none,
			domain=\base:upperboundofd1(5/6)),
			samples=60,samples y=0,
			line width=\mylinewidth
		]
		(
		{5/6},
		{0},
		{x}
		);


		% ###########################
		% Middle between upper and lower

		\addplot3[ccenter,
 				domain=1/2:1, y domain = 0:0,
 				line width=\mylinewidth]
		(
		{x},
		{0},
		{\base + radiusadmissible(x)}
		);


		% ###########################
		% Half circles admissible

		\addplot3 [black!60!white, name path = zerothcircle, mark=none, domain=0:180,samples=60, samples y=0, line width=\mylinewidth]
		(
		{1/2},
		{radiusadmissible(1/2)*sin(x)},
		{\base + radiusadmissible(1/2) + radiusadmissible(1/2)*cos(x)}
		);

		\addplot3 [black!60!white, mark=none, domain=0:180,samples=60, samples y=0, line width=\mylinewidth]
		(
		{4/6},
		{radiusadmissible(4/6)*sin(x)},
		{\base + radiusadmissible(4/6) + radiusadmissible(4/6)*cos(x)}
		);

		\addplot3 [black!60!white, mark=none, domain=0:180,samples=60, samples y=0, line width=\mylinewidth]
		(
		{5/6},
		{radiusadmissible(5/6)*sin(x)},
		{\base + radiusadmissible(5/6) + radiusadmissible(5/6)*cos(x)}
		);

		% ###########################
		% Fill

		% between zeroth circle and z-axis
		\addplot [black, opacity= 0.4, draw = none]
		fill between[of = zerothvertical and zerothcircle, reverse = true];

		\addplot [black, opacity = 0.4, draw = none]
		fill between[of = upper and lower, reverse = true];


		% ###########################
		% Points

		% middle
		\addplot3[ccenter,
				pointsinoverviews,
				domain=1/2:1,
				samples=4
				]
		(
		{x},
		{0},
		{\base + radiusadmissible(x)}
		);

		% top
		\addplot3[ctop,
				pointsinoverviews,
				domain=1/2:5/6,
				samples=3
				]
		(
		{x},
		{0},
		{upperboundofd1points(x)}
		);

		% bottom
		\addplot3 [cbottom,
				pointsinoverviews,
				domain=1/2:5/6,
				samples=3
				]
		(
		{x},
		{0},
		{\base*(9/10)}
		);



	\end{axis}
\end{tikzpicture}



\end{document}